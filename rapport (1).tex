\documentclass[9pt, french]{article}
\usepackage{graphicx}
\usepackage[T1]{fontenc}
\usepackage[utf8]{inputenc}
\usepackage{lmodern}
\usepackage[a4paper]{geometry}
\usepackage{babel}
\usepackage{wrapfig}
\usepackage[figurename=Fig.]{caption}
\usepackage{subfig}
%\title{ \textbf{} }
%\author{
%	Vilon Saint-Fleurose \\
%	Département Informatique et Recherche Opérationnelle\\
%	Université de Montréal\\
%	vilon.saint-fleurose@umontreal.ca
%}
%\date{\today}

\begin{document}
%	\maketitle

	\begin{titlepage}
		
		\newcommand{\HRule}{\rule{\linewidth}{0.5mm}} % Defines a new command for the horizontal lines, change thickness here
		
		\center % Center everything on the page
		
		%----------------------------------------------------------------------------------------
		%	HEADING SECTIONS
		%----------------------------------------------------------------------------------------
		
		\textsc{\LARGE Université de Montréal}\\[1cm] % Name of your university/college
		\textsc{\Large Faculté des arts et des sciences}\\[0.5cm] % Major heading such as course name
		\textsc{\large Département d’informatique et de recherche opérationnelle (DIRO) }\\[0.5cm] % Minor heading such as course title
		
		%----------------------------------------------------------------------------------------
		%	TITLE SECTION
		%----------------------------------------------------------------------------------------
		
		\HRule \\[0.4cm]
		{ \huge Rapport de stage \\ \bfseries Évaluation du risque de retour à la maison } \\[0.4cm] % Title of your document
		\HRule \\[1.5cm]
		
		%----------------------------------------------------------------------------------------
		%	AUTHOR SECTION
		%----------------------------------------------------------------------------------------
		
		\begin{minipage}{0.4\textwidth}
			\begin{flushleft} \large
				\emph{Auteur:}\\
				Vilon \textsc{Saint-Fleurose} % Your name
				\\MSc, informatique \\Université de Montréal
			\end{flushleft}
		\end{minipage}
		~
		\begin{minipage}{0.4\textwidth}
			\begin{flushright} \large
				\emph{Directeur de recherche:} \\
				Dr. Michalis \textsc{Famelis} % Supervisor's Name
				\\Professeur adjoint \\Université de Montréal
			\end{flushright}
		\end{minipage}\\[2cm]
			~
			\begin{minipage}{0.4\textwidth}
				\begin{center} \large
					\emph{Superviseur:} \\
					 Nicolas \textsc{Coallier} % Supervisor's Name
					\\Vice-Président Exécutif, TIC \\ML+
				\end{center}
			\end{minipage}\\[2cm]
		
		% If you don't want a supervisor, uncomment the two lines below and remove the section above
		%\Large \emph{Author:}\\
		%John \textsc{Smith}\\[3cm] % Your name
		
		%----------------------------------------------------------------------------------------
		%	DATE SECTION
		%----------------------------------------------------------------------------------------
		
		{\large \today}\\[2cm] % Date, change the \today to a set date if you want to be precise
		
		%----------------------------------------------------------------------------------------
		%	LOGO SECTION
		%----------------------------------------------------------------------------------------
		
		\includegraphics[width=4cm, height=1.2cm]{logo.png} % Include a department/university logo - this will require the graphicx package
		
		%----------------------------------------------------------------------------------------
		
		\vfill % Fill the rest of the page with whitespace
		
	\end{titlepage}
	
	\section*{Remerciements}
	Je veux commencer d'abord par remercier le Grand Dieu Tout-Puissant, le Créateur de l'univers, des cieux et de la terre qui m'a donné la vie, la santé, les opportunités et tout ce dont j'avais besoin pour faire cette grande et belle étude à l'université de Montréal. Il a rendu toutes choses possibles en ma faveur, moi qui suis pécheur et désobéissant; immérité de toutes ces grâces. Il m'accompagnait toujours dans les moments les plus difficiles de ma vie, Il ne m'a jamais laissé seul; surtout dans les moments où je devais payer les frais de scolarité qui étaient si énormes et impossibles à payer de mon propre compte. A un Dieu si merveilleux et si bon, je Lui dois beaucoup de reconnaissance.  
	
	Je remercie aussi ma femme qui m'a beaucoup supporté pendant plus de deux années d'études. Elle n'a jamais murmuré, ni découragé quand nous devions passer par des moments difficiles de notre vie conjugale à cause de ces études. Elle a mis toutes ses ressources disponibles pour entretenir la famille et payer mes études quand j’étais moi-même dans l’impossibilité de travailler. Vraiment, ma femme est une bénédiction dans ma vie, un cadeau venant de Dieu. Je t’aime ma chérie.
	
	Je tiens à remercier le professeur Michalis Famelis d’avoir accepté être mon directeur de recherche et supervisé ce stage, il est toujours là pour m’encourager et me pousser vers l’avant. Il répond toujours présent à tous mes appels, il est toujours disponible pour me rencontrer, me parler et me conseiller ; même en dehors du cadre universitaire. 
	
	Je remercie Nicolas Coallier et toute l’équipe ML+ qui ont accepté que je sois leur stagiaire, ils ont placé  leur confiance en moi quoiqu’ils ne me connaissaient pas encore. Cette équipe, quoique jeune, est très dynamique et chaleureuse, c’est une équipe motivante qui stimule la connaissance. J’ai dû apprendre beaucoup de choses par rapport à eux.
	Finalement, je présente mes sincères remerciements à toute la communauté universitaire, à DIRO en particulier. Merci pour la formation prestigieuse que vous m’avez fournie. Cette formation est si solide qu’elle m’aidera rapidement à intégrer le marché du travail sans perdre de temps. 
	
	\newpage
	

	\section{Introduction}
	
	La science des données est un domaine interdisciplinaire de méthodes, processus, algorithmes et systèmes scientifiques permettant d'extraire des connaissances ou des informations à partir de données sous diverses formes, structurées ou non. Elle emploie des techniques et des théories tirées de plusieurs autres domaines plus larges des mathématiques, la statistique principalement, la théorie de l'information et la technologie de l'information, notamment le traitement de signal, des modèles probabilistes, l'apprentissage automatique, l'apprentissage statistique, la programmation informatique, l'ingénierie de données, la reconnaissance de formes et l'apprentissage, la visualisation, l'analytique prophétique, la modélisation d'incertitude, le stockage de données, la compression de données et le calcul à haute performance. Les méthodes qui s'adaptent aux données de masse sont particulièrement intéressantes dans la science des données, bien que la discipline ne soit généralement pas considérée comme limitée à ces données.\\
	
	L'objectif du « data scientist » (expert en données massives) est de produire des méthodes (automatisées, autant que possible) de tri et d'analyse de données de masse et de sources plus ou moins complexes ou disjointes de données, afin d'en extraire des informations utiles ou potentiellement utiles. \\
	Le métier de data scientist est apparu pour trois raisons principales :
	
	\begin{itemize}
		\item l'explosion de la quantité de données produites et collectées par les humains;
		\item 	l'amélioration et l'accessibilité plus grande des algorithmes de machine learning;
		\item l'augmentation exponentielle des capacités de calcul des ordinateurs;
	\end{itemize}	
	Le cycle de travail du data scientist comprend notamment:
	\begin{itemize}
		\item la récupération des données utiles à l'étude;
		\item le nettoyage des données pour les rendre exploitables;
		\item une longue phase d'exploration des données afin de comprendre en profondeur l'articulation des données;
		\item la modélisation des données;
		\item l'évaluation et interprétation des résultats;
		\item  la conclusion de l'étude : prise de décision ou déploiement en production du modèle.
	\end{itemize}
	Au sein de ce cycle, le machine learning désigne l'ensemble des méthodes de modélisation statistique à partir des données, et se situe bien au coeur du travail de data scientist.\\
	

	Étant stagiaire en science de données, nous avons réalisé, dans le cadre de notre projet, le cycle de travail complet d'un data scientist. Depuis la récupération des données jusqu'au déploiement d'un système en production. Nous expliquerons en détail dans la section "Solutions proposées", comment nous implémentons ce cycle au sein de notre projet. \\ 
	
	Deux composantes sont nécessaires pour pouvoir commencer à se demander si la data science peut, oui ou non, apporter de la valeur et aider à la résolution d'un problème : des \textbf{données} et une \textbf{problématique} bien définie.\\ 
	Les données constituent la ressource principale pour qu'un data scientist puisse effectuer son travail correctement. Nos données proviennent du département orthopédique de l'université de Seattle, Washington, États-Unis. Elles sont au nombre de 2718 files et 92 colonnes. Nous les utilisons comme données historiques (\textit{historical data, en anglais}) dans le cadre du développement de notre modèle de machine learning. \\ 
	La problématique de notre projet est d'aider à décider si un patient doit rester à l'hôpital ou rentrer chez lui après une intervention chirurgicale.  
	

	\section{Solutions techniques proposées}
	
	

	
	
\end{document}
